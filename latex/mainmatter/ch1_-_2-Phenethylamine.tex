%page 0017

\chapter{2-Phenethylamine}

The 2-phenethylamine (1), also known as β-phenethylamine, phenethylamine or PEA, represents the basic structure of the compounds shown in this book. When researching chemical or natural substances that have this substructure, one arrives at
to an enormous variety of substances. Representatives from natural substances in the animal and plant world, as well as from agrochemistry or medicines, and even endogenous substances have this central 2-phenethylamine substructure (see also excursus on natural substances).
The unsubstituted phenethylamine (1) is an alkaloid that is found in both the plant and animal kingdoms. Probably the most well-known representatives here are cocoa or chocolate.
It is also worth mentioning that tigers (Panthera tigris) excrete phenethylamine (1) in their urine to mark their territory [1, 2]. However, phenethylamine (1) was also found in the urine of leopards [3]; therefore phenethylamine (1) alone cannot be used to mark territories. The occurrence of phenethylamine (1) in the human body is also of importance. It is called a biogenic amine because it is produced by decarboxylation of the natural amino acid L-phenylalanine (2a) in the organism.

\includegraphics{placeholder}

According to the Beilstein CrossFire® database, phenethylamine (1) was first chemically produced in 1875 by the researchers Colombo and Spica [4]. In 1910 it could be isolated from rotten meat (putrescence), and then its activity on blood pressure was determined (summarized in [5]). It was examined intensively by the two important researchers Barger and Dal in 1910 together with a large number of other amines in a comparative study on the influence on blood pressure in cats [5]. In the case of IV administration, it is able to increase this significantly.
Another important study was carried out in 1938: Phenethylamine (1) was tested for its pharmacological properties together with amphetamine and some N-monoalkylated amphetamine derivatives [6] (see also Chapter 3.1.1.2). It turned out that, unlike amphetamine, phenethylamine (1) is unable to wake anesthetized mice. When the effects in humans were investigated, it was ascertained that phenethylamine (1) did not have any amphetamine-like effects. Shulgin leads in the standard work PiHKAL p.o. Test doses of up to 1600 mg phenethylamine (1) in humans, none of which were able to develop any effect either [7]. This is contradicted by information that was circulating on the Internet: Allegedly, dosages should be im
\clearpage

%page 0018
A range of 1 to 4 grams have a similar effect as a mixture between Δ9-THC (tetrahydrocannabinol; main active ingredient in hemp) and the entactogen MDMA (see chapter 7.5.6). This disinformation circulated only for marketing reasons, in order to boost sales of (legal) phenethylamine (1) through online shops [8]. There is also evidence of the (risky) combined intake of monoamine oxidase inhibitors (MAO inhibitors) with phenethylamine (1) [9] - the euphoric and stimulating effects are controversial, and it can be assumed that this is the main reason going to make profitable sales.
In order to shed light on the pharmacological background to the ineffectiveness, some of the studies with phenethylamine (1) described in the specialist literature are shown here
will.

It is known that central stimulating substances such as amphetamine stimulate locomotion. This is particularly evident in rats and mice. In 1962 Van der Scoot et al. this connection in a large number of phenylalkylamine derivatives [10]. In comparison to amphetamine, phenethylamine (1) was found to be completely inactive in this test arrangement (mouse, i.p. administration). In order to clarify whether the lack of an effect is due to bio-inactivation, the following experiment was carried out: It is known that substances with an amphetamine structure have good stability towards the endogenous
Show enzyme monoamine oxidase (MAO, see excursus amine oxidases). Substances that do not have an α-methyl group, namely phenethylamines, have very little resistance to this enzyme (especially those with few or no substituents on the aromatic) and are therefore rapidly broken down (Fig. 1). Now mice were first given the MAO inhibitor iproniazid to suppress the degrading effect of MAO on phenethylamines. When phenethylamine (d) i.p. was administered, they showed a very pronounced urge to move. The thus increased Lokomotion underpinned the theory of bio-inactivation of simple
Phenethylamines by MAO. This could later be confirmed several times (e.g. [11], and summarized in [12]).

In rhesus monkeys, changes in behavior were found with phenethylamine (1) after the previous administration of MAO-A and MAO-B inhibitors [13].
Other experiments have also shown that prior administration of an MAO inhibitor 1 in test animals has a spectrum of activity comparable to that of amphetamine.


\includegraphics{placeholder}

Fig. 1. Oxidative deamination of phenethylamine (1). There
Introducing Deuterium Atoms into the Alpha Position for the
Nitrogen makes the metabolic breakdown by monoamine oxidase (MAO) significantly more difficult.
\clearpage

%page 0019
The doubly deuterated α,α-dideuterophenethylamine (4) proved to be stable towards MAO (Fig. 1) [14]. It showed a long-lasting stimulus, completely generalized for phenethylamine (1) in a drug discrimination study (rats) and was clear
more potent than this [14]. It would also be interesting to test whether α, α-dideuterophenethylamine (4) would be a substitute in amphetamine-trained rats. So far there are no reports
its effect in humans is known. It could well be that this concept of the metabolic stabilization of phenethylamine derivatives can also be extended to other phenethylamines, which are in and of themselves in vivo inactive phenethylamines (see also Chapter 8.4.6, where the increase in the human potency of mescaline by an α, α - the explanation is documented).
It has been shown, for example, that phenethylamine (1) without an adjuvant in small doses gives cats a state of increased alertness and agitation, just as amphetamine can, and at higher doses phenethylamine (1) alone could also have effects like amphetamine, although the The duration of the effects was shorter (probably as a result of autoinhibition; summarized in [12]). It has also been shown that phenethylamine (1) can maintain self-administration in dogs [15]. However, there are still doubts as to whether phenethylamine (1) can even produce a stimulatory effect in humans after reaching the brain due to its metabolic lability. Binding properties to monoamine transporters, the actual targets for stimulants such as amphetamine, have apparently been completely absent to date [32].
Vaupel et al. presented in their study that the administration of phenethylamine (1) to dogs shows some amphetamine-like effects [16]. It was also compared whether, for example, breathing and pulse frequency, pupil diameter and body temperature
behave in the same way in dogs treated with LSD compared to untreated dogs.
S-amphetamine showed no cross-tolerance to LSD, and phenethylamine (1) also showed only a very weak cross-tolerance to LSD, which in view of the different targets (dopamine and norepinephrine transporters for amphetamine, and above all various 5-HT and dopamine Receptors for LSD) is not surprising either.

Nakajima, Kakimoto and Sano discovered the natural occurrence of phenethylamine (1) in the mammalian brain in 1964 [17]. Phenethylamine (1) was later declared a biogenic amine (summarized in [18]). L-phenylalanine (2a) is converted into phenethylamine (1) in the capillary endothelium of the brain and in catecholaminergic nerve endings by a simple decarboxylation (Fig. 2) [19]. The conversion takes place at about the same speed as the central synthesis of dopamine. Since phenethylamine (1) is not stored in intraneuronal vesicles, it is not considered a classic neurotransmitter, but is nonetheless an endogenous ligand on the trace along with other biogenic amines.
Amine receptor [20]. Like the other biogenic amines, it is broken down very quickly by the MAO, especially the MAO-B, and thereby inactivated [21].
It has been shown that the concentration of natural phenethylamine (1) in the brain increases sharply if an MAO inhibitor is administered beforehand (summarized in [22]).
There were also indications that phenethylamine (1) could be a neuromodulator of catecholamine activity. Karoum et al. investigated this on the amphetamine-induced release of dopamine [23]. Their results suggested that in addition to the release of dopamine after the amphetamine administration, phenethylamine (1) was also formed and released. However, these two processes are not directly related, and they assumed that the two substances are not released or formed at the same time. Thus, the effect of phenethylamine (1) should not be dependent on the release of dopamine.
\clearpage

%page 0020
\includegraphics{placeholder}
Fig. 2. Simplified representation of the biosynthesis of phenethylamine (1) and the subsequent metabolism (see also "Biological origin of phenethylamine" in the excursus on natural substances).


It has been argued that too low brain phenethylamine concentrations could be a cause of depression [24]. To test the hypothesis that phenethylamine (1) modulates mood disorders, the concentration of phenylacetic acid (,, the main metabolite of phenethylamine (1), in the plasma and urine of depressed persons [24]. The concentrations were compared to healthy persons This suggested that low plasma and urinary phenylacetic acid concentrations could serve as a diagnosis of depression. They also showed that administration of the phenethylamine precursor L-phenylalanine (2a) had a positive effect on mood in 31 of 40 patients would.
In a further experiment, 10-60 mg phenethylamine (1) was administered orally daily for 20-50 weeks, with 10 mg selegiline being released at the same time in order to counteract the rapid breakdown of phenethylamine (1) by MAO [25]. In 12
of 14 patients, phenethylamine (1) showed an antidepressant effect. This supported the authors' hypothesis that deficient phenethylamine concentrations could be directly related to depression. The administered dose of phenethylamine (1) remained constant in the test subjects. This showed that, in contrast to amphetamine, there was no development of tolerance with phenethylamine (d). However, it remains to be added that the sole effect of selegiline has not been tested in a control group and that today selegiline is also used to treat depression in individual countries. This currently calls into question the informative value of the small-scale study.

Interestingly, the unnatural enantiomer D-phenylalanine (2b) also shows pharmacological effects. Antidepressant [26], but also analgesic [27] effects are ascribed to it. Which mechanism or which targets are involved is ultimately not clear. The enzyme carbopeptidase, which also breaks down endorphins, appears to be
\clearpage

%page 0021
being inhibited by it [28]. D-amino acids are often not recognized by the actual targets (enzymes, transporters and other proteins), which leads to abnormal pharmacodynamics. The most relevant interaction proteins for an effect of an amino acid in the CNS are the aromatic amino acid transporters, which actively transport the essential aromatic amino acids into the brain, and the aromatic amino acid decarboxylase, which vitamin B6 requires as a cofactor [29].

\includegraphics{placeholder}

Despite the very short half-life (due to the rapid degradation by MAO), phenethylamine (1) has received a great deal of attention as an endogenous amine. The reason for this lies in its ability to potentiate catecholaminergic neurotransmission [19]. Fuxe et al. found that phenethylamine (1) releases dopamine and norepinephrine from presynaptic endings [30]. This means that it acted as a releaser or reverse transporter in your test system, similar to amphetamine. The increase in motor activity in test animals treated with an MAO inhibitor was ascribed to the increased release of dopamine [31].

Finally, a popular scientific and rather critical note on phenethylamine (1): In some places it can be read that phenethylamine (1), which actually occurs in cocoa products, creates feelings of happiness and is addictive when eating chocolate. However, whether the amounts of phenethylamine (1) contained are really pharmaceutical
are macologically relevant is rather doubtful.


Literature


[1] R.L. Brahmachary, J. Dutta, Z. Naturforsch. 1979, 34C, 632.
[2] J. Dutta R. L. Brahmachary, Amer. Natu. 1981, 118, 561.
[3] R. L. Brahmachary, J. Dutta, Tigerpaper 1984, 11, 18.
[4] Colombo, Spica, Gazz. Chim. Ital. 1875, 5, 124.
[5] G. Barger, H. Dale, J. Physiol. 1910, 47, 19
[6] E. Jacobsen, A. Wollstein, J. T. Christensen, Klin. Wschr. 1938, 17, 1580.
[7] A.T. und A. Shulgin, PiHKAL - A chemical love story, Bergamon Press: Berkeley CA,
USA, 1991.
[8] http://www.bluelight.ru/vb/showthread.php?t=383096, 2008.
[9] http://www.avantlabs.com/forum/index.php?showtopic=21081, 2006.
[10] J. B. Van der Schoot, E. ]. Ariens, J. M. van Rossum, J. A. Hurkmans, Arzneim. Forsch.
1962, 12, 902.
[11] M.R. Vasko, M. P Lutz, E. E Domino, Psychopharmacologia 1974, 36, 49.
\clearpage

%page 0022
[12] J. M. Saavedra, BE. Fischer, Arznem. Forsch. 1970, 20, 952.
[13] RA. Staub, J.C. Gillin, R. ). Wyatt, Bio. Psychiatry 1980, 15, 429.
[14] D. Reid, A. ]. Goudie, Pharmacol. Biochem. Behau. 1986, 24, 1547.
[15] M.E. Risner, B. BE. Jones, Pharmacol. Biochen. Behau. 1977, 6, 689.
[16] D.B. Vaupel, M. Nozaki, W. R. Martin, 1. D. Bright, Eur. J. Pharmacol. 1978, 48, 431
[17] T. Nakjima, Y. Kakimoto, I. Sano, J. Pharmacol. Exp. Ther. 1964, 143, 319.
[18] ME. Wolf, A. D. Mosnaim, Gen. Pharmacol. 1983, 14, 285.
[19] P. A. Janssen, ]. E. Leysen, A. A. Megens, EH. Awouters, Int. J. Neuropsychopharmacol.
1999, 2, 229.
[20] L. Lindemann, C. A. Meyer, K. Jeanneau, A. Bradaia, L. Ozmen, H. Bluethmann,
B. Bettler, ].G. Wettstein, E. Borroni, J.L. Moreau, M. C. Hoener, J. Pharmacol. Exp,
Ther. 2008, 324, 948.
[21] J. Bergman, S. Yasar, G. Winger, Psychopharmacology 2001, 159, 21.
[22] P.R. Paetsch, A. J. Greenshaw, Neurochem. Res. 1993, 18, 1015.
[23] F Karoum, M, E. Wolf, A. D. Mosnaim, Am. J. Ther. 1997, 4, 333.
[24] H.C. Sabelli, J. Fawcett, R Gusovsky, J. 1. Javaid, P. Wynn, J. Edwards, H. Jeffriess,
H. Kravitz, J. Clin. Psychiatry 1986, 47, 66.
[25] H. Sabelli, P. Fink, ]. Fawcett, C. Tom, J. Neuropsychiatry Clin. Neurosci. 1996, 8, 168.
[26] J. Mann, E. D. Peselow, S. Snyderman, $. Gershon, Am. ]. Psychiatry 1980, 137, 1611.
[27] A.L. Russell, M. E McCarty, Med. Hypotheses 2000, 55, 283.
[28] D. W. Christianson, $. Mangani, G. Shoham, W. N. Lipscomb, J. Biol. Chem. 1989,
264, 12849.
[29] G.F.  Combs, The Vitamins, Academic Press: San Diego, 2008.
[30] K. Fuxe, H. Grobecker, ]. Jonsson, Eur. J. Pharmacol. 1967, 2, 202.
[31] C. H. Vanderwolf, T. E. Robinson, B. A. Pappas, Brain Res. 1980, 202, 65.
[32] PDSP. online K-Database NIMH. http://pdsp.med.unc.edu/; 30.10.2012.
\clearpage

%page 0023
\begin{center}
	Excursus:\\
	Natural substances with a basic phenethylamine structure
\bigbreak

Diffuse definitions
\bigbreak
\end{center}

Natural substances are chemical substances that are obtained from living or inanimate nature. The term nature and its social interpretation is related to historical epochs and fills entire books. The dictionary defines nature as follows [1]: “The totality of organic and inorganic phenomena that exist or develop without human intervention. Matter, substance, matter in all manifestations.”

According to this definition, crude oil is just as much a natural substance as the delicate rose scent components or snake venom. The definition is already vague at this level. The definition of the term "alkaloids" is even more problematic. For example, it uses attributes such as “naturally occurring, not commonly distributed compounds containing one or more nitrogen atoms in the molecule” [2]. The group owes its name, derived from “alkaline-like” [2], to the basic nature of many of its representatives. In other definitions, only plant ingredients are even referred to as alkaloids. There are more exceptions than rules: for example, animal ingredients, non-basic alkaloids with an acid amide structure, quaternary ammonium compounds, lactams and N-oxide derivatives are also listed as alkaloids. Differentiation from other N-containing natural substances is difficult: biogenic amines, pyrazines, pterins, vitamins and their derivatives, as well as amino sugars and antibiotics are generally not counted among the alkaloids. Finally, special cases determine the classifications, which are ultimately arbitrary and, depending on the author and the state of research, more or less broad.

The number of alkaloids described to date is over 12,000 [2]. Alkaloids are often biologically active, which has been used for thousands of years. As is the case with opium, which is one of the oldest medicines [2], although here the borderline between stimulants and stimulants is fluid. Coca, tea, cocoa and tobacco have been used as stimulants since prehistoric times [2].

Whether arrow poison or medicine, mixtures of active ingredients from plants, literally "drugs", were usually used. The term "drug" as a designation for pharmaceutically active substances comes etymologically from the Dutch (drog, dt.: dry). The dictionary defines it as follows: "Plant or animal substance preserved by drying, which is used as a medicine, spice and for technical purposes" [1].  

\bigbreak
This excursus deals with substances that were created without human influence and that contain a phenethylamine structure. Many other possible connections that arise through the metabolism in humans, animals or plants have so far been little or not at all researched. The focus is on substances from nature that have an aromatic ring.
\clearpage

%page 0024
The term "phenethylamine" is vague, but the question arises whether it refers to derivatives that are similar to 2-phenethylamine. The term "same" is again dependent on the point of view. The spectrum of phenethylamines under consideration consequently expands massively the more substitutions are included: If one considers N-substituents, the question arises of rigidized derivatives such as e.g. B. the tetrahydroisoquinolines, an enormously widespread class of natural products. To a certain extent, a modified phenethylamine structure is also found in the indoles and thus the tryptamine derivatives, and finally even in morphine and LSD.
\bigbreak

Even more derivatives can be included with the term “arylalkylamines” because it also includes, for example, all the compounds with a benzylamine or 3-arylpropylamine structure. Perhaps the term “(hetero) arylethylamine” would be a little more precise. In colloquial terms, however, “phenethylamine” has simply caught on, and this primarily includes compounds that are similar to phenethylamine and amphetamine and all their derivatives.
\bigbreak

The branches (polycycles, chain length, etc.) are manifold, so the focus is primarily on substances with an ethyl unit, clamped between an aromatic or heteroaromatic and an amino group. The variation in aromatics is the most important; heteroaromatic substances have been shown to represent their independent classes.
\bigbreak

\begin{center}
Biological origin of the phenethylamine:\\
Praise to the shikimic acid
\bigbreak
\end{center}

Although mammals - and therefore humans as well - need considerable amounts of phenethylamine derivatives and these, as hormones and neurotransmitters, play a central role in the protein balance, we are still unable to produce phenyl rings build up in the body. The aromatic amino acids are counted among the essential 'amino acids, because they are our source of aromatics. But where do the phenyl rings in our food come from? They are all of vegetable origin [3, 4]. So we either have to eat the plants ourselves or the animals that ate the plants before. The plants have a refined biosynthetic pathway from which the whole living world benefits: the shikimic acid path, the starting point for the production of shikimic acid (1) is the C3 body phosphoenolpyruvate (2, PEP), which comes from glycolysis, and the C4 body erythrose-4-phosphate (3) from the pentose phosphate pathway. The following diagram shows how a cyclic hemiacetal with seven carbons (5) is formed from the C3 and C4 bodies, which forms the cyclohexene ring of shikimic acid (9) via phosphate cleavage and rearrangement.
\clearpage

%page 0025
\begin{center}\includegraphics{placeholder}\end{center}

The shikimic acid (1) is now converted to the phosphate ester 8, which then reacts with another molecule of PEP (2) to form horismic acid (9). The phosphate ester formed in 8 serves as a leaving group. The chorismic acid (9) itself is also subject to an intramolecular Claisen rearrangement and forms prephenic acid (10) through chorismate mutase.

\begin{center}\includegraphics{placeholder}\end{center}
\clearpage

%page 0026
Elimination and decarboxylation reactions then form α-keto-phenylacetic acids (11 and 13), which are converted into the respective amino acids L-tyrosine (12a) and L-phenylalanine (14a) via aminotransferases in the plant. Ultimately the phenethylamine can now be formed by decarboxylation of 14a (15; see also Chapter 1). The chorismic acid (9) can also react with glutamic acid to form anthranilic acid, which in turn can be converted into tryptophan in several steps (not shown in the scheme). All atoms drawn in gray have their origin in erythrose-4-phosphate (3), the black atoms come from phosphoenolpyruvate (2).

\begin{center}\includegraphics{placeholder}\end{center}

With this we establish that all aromatic alkaloids and every molecule of serotonin or dopamine (29), all of which contain a phenyl or an indole ring, have their origin in shikimic acid (1)!

The number of pure phenethylamines is relatively small compared to heterocyclic derivatives with an arylethylamine basic structure. This is presumably because most naturally occurring substituents on the arene are HO groups, making the arene electron rich and favoring ring closure after the amines have reacted with a carbonyl compound (aldehydes or ketones). Thus, most phenethylamines, if they are electron-rich in nature, react further in the plant to form isoquinoline derivatives.

\begin{center}
Prominent and lesser known naturally\\
occurring phenethylamines
\end{center}
\bigbreak

The naturally occurring phenethylamines are briefly portrayed below, following the degree of aryl substitution. In addition, an overview of the isoquinoline and indole derivatives provides an insight into the (incomplete) classifications. Literature references were largely dispensed with where the information relates to the Chapter in this book. It can therefore be found in detail in the chapters to which reference is made at the relevant points.

\clearpage

%page 0027
\includegraphics{placeholder}
Phenethylamine (see Chapter 1) The biogenic amine phenethylamine (2-phenethylamine or ß-phenethylamine) is found in cocoa beans, in the brains of humans and animals and in the urine of big cats. Phenethylamine (15) is formed by decarboxylation of
Phenylalanın (14) and is rapidly degraded by monoamine oxidase (MAO)-B. In the brain, like amphetamine, it is able to act as a norepinephrine and dopamine releaser on monoamine transporters (29) without having a stimulating effect in vivo—probably because of its metabolic lability. Antidepressant properties are attributed to it. It thus functions as a modulator, but not as a neurotransmitter.
\bigbreak

\includegraphics{placeholder}
L-Phenylalanine (see Chapter 1) The essential amino acid is found in vegetable and animal protein-rich foods. It cannot be made by mammals themselves and comes from the previously mentioned shikimic acid pathway of plants. L-phenylalanine (14a) is the biochemical precursor to the neurotransmitters dopamine (29), norepinephrine (31) and epinephrine (32), but is also important in the synthesis of proteins. The naturally occurring L-phenylalanine (14a, S-phenylalanine) tastes bitter, while the synthetic D-phenylalanine (14b, R-phenylalanine) leaves a sweet taste impression. Interestingly, the unnatural D-phenylalanine (14b) itself also shows pharmacological effects. Antidepressant effects are also ascribed to it (see also [5]). In the liver, L-phenylalanine can also be hydroxylated on the aromatic, which creates the amino acid tyrosine (12a).
\bigbreak

\includegraphics{placeholder}
Ephedrine (see chapter 3.5) The ephedra alkaloids occur in several plants. The most well-known representative is the family of the seaweed plants (Ephedraceae), from which the name is derived. It is also found in the blue monkshood (Avonitum napellus), the yew (Taxus baccata) and in the Kath shrub (Catha edulis). Because of of the two stereocenters there are four stereoisomers. Their affinities for adrenergic α and ß receptors are relatively weak.
They act primarily as releasers and reuptake inhibitors on the norepinephrine transporter and thus, as indirect sympathomimetics, have blood pressure-increasing, heart-stimulating, bronchodilator, stimulating and appetite-suppressing properties [2]. The synthetic ephedrine stereoisomer 16b is less potent than the naturally occurring 16a. Of the pseudoephedrines, the (+) - [18,28] form 16c occurs naturally, while the (-) - [1R, 2R] form 16d is obtained synthetically.
\clearpage

%page 0028
\includegraphics{placeholder}
Norephedrine (see chapter 3.5) Removing the N-methyl group of the four stereoisomers of ephedrine leads to the four stereoisomeric norephedrines 17a-d (the prefix Nor means N without a remainder). 17a is referred to as (-norephedrine, 17b as (+) - norephedtine, 17c as (+) - cathine or (+) - pseudonorephedrine and 17d as (-) - cathine or (-) - pseudonorephedrine.

\includegraphics{placeholder}
N-methylephedrine (18) is also found naturally in ephedra. It has been tested as a broncholytic agent and is used today for colds and as a treatment for coughs.

\includegraphics{placeholder}
S-cathinone (19, see Chapter 3.5) occurs mainly in the leaves and young twigs of the Kath shrub (Catha edulis).
The mode of action is similar to that of ephedrine (46a). On the Arabian Peninsula and the Horn of Africa the leaves are
and the young twigs used as intoxicants. Increased libido and willingness to perform as well as euphoria and talkativeness are typical effects. In the case of habitual cath abuse, apathy prevails [2]. It was only discovered in 1981 that cathinone (19) is the actual active principle for the stimulating effect of the plant. This late discovery may have been due to common isolation practices. During the extraction process the relatively unstable α-aminoketone comes into contact with acids, bases and atmospheric oxygen and dimerizes to a diphenylpyrazine (20) [2].

\includegraphics{placeholder}
Tyrosine and its regioisomers. The naturally occurring L-tyrosine (12a; L-Tyr, see Chapter 6.3.1) is a non-essential amino acid for humans, as it can be produced from the essential L-phenylalanine (14a). It was isolated from cheese by Justus von Liebig in 1860 (Greek Tpog, pronounced fyros, German: cheese). L-Tyr (12a) is almost tasteless or slightly bitter, whereas the synthetic D-Tyr (12b) tastes more sweet.

In addition to L-tyrosine (12a), the regioisomers zeta- and ortho-tyrosine (21 and 22) occur in nature. They arise from L-phenylalanine (14a) by hydroxylation with free radicals under conditions of oxidative stress [6]. As described in chapter 7.5.1
\clearpage

%page 0029
\includegraphics{placeholder}tyrosine (12a) can be both decarboxylated and hydroxylated on the aromatic ring. Thus, it is a precursor to tyramine (23), dopamine (29), norepinephrine (31), and epinephrine (32).

\includegraphics{placeholder}
Tyramine (see Chapter 6.3.1) As a decarboxylation product of tyrosine (12a), tyramine (23) is found in many plant and animal foods: The most prominent representative is also the namesake: cheese (gr. ?????, pron. tyros, dt .: cheese), chocolate, beer, red wine, salami, soy products, yeast extract, avocados and many more. The biogenic amine tyramine (23) is transported into the cell interior by the norepinephrine transporter (NET), where it is quickly broken down by the MAO and therefore has almost no effect in humans. In combination with MAO inhibitors, it acts as an indirect sympathomimetic and thus releases norepinephrine in the periphery, which can lead to severe headaches and even a life-threatening increase in blood pressure (see also the excursus on amine oxidases). One speaks of the “cheese effect”. Tyramine (23) is also an endogenous ligand at the trace amine receptor (TAAR). In insects it is considered a messenger substance in the nervous system.

\includegraphics{placeholder}
Leptodactyline (24, see Section 6.2.3.3) has been identified in extracts from fresh or dried amphibian skins from over 140 species (not bufonids). Leptodactylidae is the name given to a widespread family of frogs (Anura). This zwitterionic phenethylamine acts as a muscle relaxant at the nicotinic acetylcholine receptor and is not available orally. Octopamine (25, see Chapter 6.3.1) has been isolated in the salivary glands of octopuses. It acts as a neurotransmitter and modulator in the nervous system of insects, spiders, snails and crustaceans. It is also advertised online as a diet drug because of its sympathomimetic effects. Synephrine (26, see Chapter 6.3.1) is found naturally in various plants, above all in citrus species, and especially in the bitter orange (Citrus aurantium). A number of entries on the Internet tout synephrine (26), similar to ephedrine, as a performance-enhancing and fat-burning agent. It acts as an α-sympathomimetic and as a ß3-agonist [7, 8]. Hordenine (27, see Chapter 6.3.1) occurs in various plants. Similar to ephedrine and tyramine (23), it shows a blood pressure-increasing effect. In locusts it acts as a feeding inhibitor [52].
\bigbreak

Chloramphenicol (28) is a widely used antibiotic that has been in use since the 1950s. It was isolated from the bacterium Streptomyces venezuelae and is now rarely used due to its side effects (possible bone marrow depression) [9].
\clearpage

%page 0030
\includegraphics{placeholder}
In addition, chloramphenicol (28) is one of the few natural products that contain a nitro group.

\includegraphics{placeholder}
Dopamine (DA, 29, see chapter 7.5.1). The name derives
from dihydroxy-phenethylamine. The neurotransmitter off
the group of catecholamines occurs in different brain areas of mammals. The substanceia nigra (black substance) is particularly rich in DA and has its color because DA with its ortho-dihydroxy structure is very sensitive to oxidation and easily transforms into quinoid systems that turn dark.
DA binds to five receptor subtypes (D1-D5) and regulates many processes in the brain that play crucial roles in human mental and physical health. It influences emotions such as joy, enjoyment or lust and is therefore also referred to as the messenger substance of the reward system. DA not only influences emotional processes in the CNS (middle, intermediate and cerebrum), it also coordinates movement sequences and is responsible for the blood flow to internal organs in the autonomic nervous system. Various clinical pictures such as schizophrenia, Parkinson's, erectile dysfunction or restless legs syndrome are associated with DA. DA is also related to addictive behavior: stimulants such as amphetamines achieve their effects primarily by releasing DA (and norepinephrine), which is why they are also known as indirect dopamine agonists.
This justifies the addiction potential of these substances. DA is also indirectly held responsible for the dependence on opioids, benzodiazepines, nicotine and others. It is an intermediate in the biosynthesis of norepinephrine (norepinephrine; 3) and Epinephrine (41).
\clearpage

%page 0031
L-DOPA and the Pathways of Melanins. The amino acid L-DOPA (30, see chapter 7.5.1) plays a central intermediate stage in catecholamine biosynthesis. It is still the most important drug for the treatment of Parkinson's disease to this day.
L-DOPA (30) also plays a key role in skin pigmentation. As can be seen from the following diagram, via DOPA quinone (33) on the one hand dopachrome (34) and on the other hand thio derivatives (35 and 36) of the L- DOPA (30), which cyclize to benzothiazines (37 and 38). Polymers that arise from it are called
\includegraphics{placeholder}
\clearpage

%page 0032
\clearpage

%page 0033
\clearpage

%page 0034
\clearpage

%page 0035
\clearpage

%page 0036
\clearpage

%page 0037
\clearpage

%page 0038
\clearpage

%page 0039
\clearpage

%page 0040
\clearpage

%page 0041
\clearpage

%page 0042
\clearpage

%page 0043
\clearpage

%page 0044
\clearpage

%page 0045
\clearpage

%page 0046
\clearpage

%page 0047
\clearpage

%page 0048
\clearpage


