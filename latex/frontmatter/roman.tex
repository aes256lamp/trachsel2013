
\section{Foreword by David E. Nichols}

%page I
Foreword by David E. Nichols
When Daniel Trachsel asked me to write a few introductory words to a new book on phenethylamines, I gladly agreed, thinking it would be an exciting task and probably not difficult. When the raw manuscript was presented to me for review, I was completely surprised. I immediately wrote back and asked, "How long did it take you to do this work?" I asked because the book is nearly 1000 pages long, describes more than 2300 compounds, and is backed up by more than 2400 literature references!

The title of the book is deceptively simple and refers to the content, which describes how the individual structural elements of phenethylamines lead to their pharmacological effects. But this is no ordinary book. In fact, I cannot recall ever having seen such a comprehensive work describing all variants of phenethylamines along with their wide range of biological effects. There are ten main chapters, each divided into various sub-chapters.
Each chapter focuses on a specific substitution pattern on the phenethylamine core. The chapters not only contain structure-activity relationships, but also often discussions on historical aspects: When was which compound developed by whom in which context, and what ultimately became of the compound when it was intended for therapeutic use? A large number of interesting advertisements from old pharmaceutical companies supplement the scientific explanations and thus show the social significance of these connections.

The extent to which the authors have searched the literature to uncover information on older compounds is remarkable. It must have been extremely time-consuming to collect the many old literature references and historical advertisements from pharmaceutical companies. To give typical examples, Chapter 3 focuses on phenethylamines without aromatic substituents, but with various substituents in the side chain or on the nitrogen. Another chapter deals with "phenethylamines" in which the phenyl ring is replaced by a variety of heteroaromatic ring systems. There are chapters that cover phenethylamines with mono-, di-, tri-, tetra-, and penta-substituted ring systems, respectively that the same compound is discussed in more than one chapter, appropriate cross-references are provided A visual table of contents at the beginning of each chapter shows the structures discussed within a chapter, making it easy for the medicinal chemist to quickly find the structures of interest All shown Structures are accompanied by pharmacological data.The data for older compounds are often derived from animal or even human studies and are representative of pharmacological techniques used decades ago.Where newer compounds are discussed, the pharmacological data relate to modern techniques. there In addition, so-called digressions are included in many chapters. These digressions are comprehensive historical profiles that place the explanations in a broader context. For example, Chapter 1 contains
\clearpage

%page II
an excursus on natural products that carry a phenethylamine backbone and phenethylamine biosynthesis via the shikimic acid pathway is presented therein as well as a discussion on essential amino acids and neurotransmitters. Other digressions, to name a few, cover areas such as antidepressants, monoamine transporters and anorectics for obesity. The digressions are consistently very interesting and provide historical details that are no longer familiar to a scientist today. The digression on anorexics gives an example of how diet-obsessed western society is and how phenethylamines of various kinds are brought onto the market without further ado, but ultimately failed due to a lack of effectiveness in weight reduction. The immense addictive potential of such connections is described; but there is also a nice discussion of the phenomena occurring with the use of the fen/phen combination, which ultimately led to the finding that 5-HT2B agonists can cause pulmonary hypertension and valvular heart disease.

Other chapters cover ring-substituted hallucinogenic (psychedelic) compounds, beginning with the isolation and structural elucidation of mescaline, the naturally occurring prototype of this class of compounds. I am particularly familiar with these compounds, and this book presents one of the most comprehensive reviews of hallucinogenic phenethylamines I have ever seen.

This book is a valuable addition to the bookshelf of any medicinal chemist with a deep interest in phenethylamines and, more generally, in centrally active compounds. The only negative I have to say about this book is that it is only published in German. I believe it would find a wide readership among English-speaking medicinal chemists and pharmacologists, and it is to be hoped that the demand for this book will be sufficient for an English translation to appear soon.

David E Nichols, PhD
October 26, 2012
Chapel Hill, NC*


* translation of the foreword by David E. Nichols by the authors.
\clearpage

%page III
\section{Foreword by the authors}
Phenethylamines - behind this word hides a large family of multifunctional chemical compounds that are omnipresent in our everyday life. A privileged and much-studied structure, starting with agents that act and interact in our body, and not infrequently in the brain. In the periphery, epinephrine (adrenaline) controls our escape behavior, and since the human psyche forms a sophisticated communication system, important neurotransmitters with a phenethylamine structure such as dopamine and norepinephrine also accompany us. In plants we find natural substances such as cathinone, mescaline and ephedrine, which have been used by humans for a wide variety of purposes for thousands of years.

Due to this distribution in nature and the similarity with our messenger substances, they offer extremely valuable templates for the development of drugs. For about a century, they have inspired scientists to create synthetic analogs from a variety of motifs using the technical capabilities available to them. Methamphetamine, one of the first man-made derivatives of phenethylamine, was discovered in the late 1800s and commercialized as a drug decades later. Other important representatives followed and influenced our culture in groundbreaking ways. Over the course of a century, scientists created analgesics, psychedelics, antidepressants, anorexics, antipsychotics, and stimulants based on phenethylamine—drugs that were needed and later often abused. A cycle that was repeated with prominent representatives such as MDMA (ecstasy) and anorexic drugs (phentermine) and continues to this day. Great importance must be attached to this influence, and not only that of psychedelics on the music and art scene - the social impact of phenethylamines ranges from (temporarily) happy, fast and slimming pills to stimulants during wars. Due to their mind-altering potential, they serve not only as "intoxicants", but also as research tools (e.g. DOI, DOM) to study neural processes and interactions - in the hope of getting a step closer to the mystery of the human psyche.

With around 2300 phenethylamine derivatives, the vast majority of the compounds known today were portrayed in this book; However, since new derivatives from research and development are constantly being added and this process — albeit at a slower pace — is far from over, it remains a comprehensive snapshot. This book has set itself the task of classifying and presenting the deep connections and relationships between the individual representatives.

In the world of the Internet with its forums and blogs, in addition to the scientific inconsistencies of swarm intelligence, there is also swarm selection - some things are inevitably overrated, others neglected. Quantity and quality remain without verifiable criteria. What good is knowledge if it is not placed in a context? In a society in which knowledge is created faster and faster, it is not enough to have just one
\clearpage

%page IV

\clearpage

\section{Table of Contents}
%	\tableofcontents

\section{Visual table of contents}
picture
picture
\clearpage

\section{Introduction}
blah blah blah
\clearpage

\section{Navigation and Construction}
fjdkl
fdsjkl
fsdjl
\clearpage

\section{Introduction to the structure-activity relationships of phenylalkylamines}
ljfsd
fdsjlfs
